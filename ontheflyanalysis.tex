\section{On-the-fly analysis of distributed computations \\
\small{Eddy Fromentin, Claude Jard, Guy-Vincent Jourdan, Michel Raynal}}

\subsection{Overview}

This paper proposed a method for checking properties of distributed
computations during exectuon. The Idea is a bit suble, and very general. Given
a standard model of a distrubuted systems ie $P_0 \dots P_n$ processand with
events $e_0 \dots e_m$ apply to each event a work or subset of an alphabet
$\Sigma$. As control flow passes through events on different machines check the
current alphabet againts a specified automata or regular expression. If an
event fails to correlate with the regular expression then it has not met the
specification and fails. Otherwize the letters of the event are appended to the
message, and upon the next event it is checked againts the regex again.

There is a little bit in the beginning about simplifing a distributed execution
down to an LPO. The graph (Figure 1b) does not make a lot of sense. It seems to
trim out long messages, and only be concerned with the longest message path
through the execution. It was not totally clear to me why this would be used,
but I guess the idea is that newer messages have some sort of precident, and
can invalidate old messages??


