\section{Some Optimal Algorithms for Decomposed Partially Ordered Sets \
\small{ Vijay K. Garag}}

\subsection{Notes}

I'm reading this one as both an introduction to Garag and as a bit of a break
from the proof heavy papers.

This paper allready seems like it is going to be a good resource, it has an
algorithm for detecting consistant cuts $n^2m$ time $\dots$ if this is really
the case, it may be possible just cut the crap on dinv and really speed it up.


As usual my hopes are dashed this is an early algorithm which only detects a
single consistant cut, not all of them like I had wanted. However, the online
variant may be useful for making \dinv a streaming service at some point.

The second algorithm in this paper checks if a set of given antichains compose
are constrained by a total ordering. The algorithm does so in $O(mn log n)$
time. Essentially it continuously merges sets of antichains untill it finds
aconflits where some set of antichains has events / elements $s$ and $t$ such
that $s || t $. It is a very nice algorithm, but given that \dinv never
encounters total orderings I don't think that it will come in handy.

\subsection{Observations}

The online detection of consistant cuts could be used as a fron end for dinv
easily, but the cost of checking all consistant cuts is still exponential.

\subsection{Questions}

\begin{itemize}

\item What is the purpose of finding largest antichains? Potentially it shows that the largest number of chains which can be composed is $n$ but it is still not clear where this would be applied. Is this to show the complexity during distributed debuggin?

\end{itemize}
