\section{Specification and verification of Dynamic Properties in Distributed Systems \\ \small\textit{Ozalp Babaoglu and Micheal Raynal ISIRA rennes}}

\subsection{Notes} interesting point early on that any number of events may be
"irrelevent with respect to a given property" so their model simply ignores
all that are not relevent. In the future consider providing a feild or
annotation in a messaging block, (such as govector) which would allow the
isolation of messages which pertained to a given property. This would allow
for a higher degeree of fidelity. In addition it may be possible to associate
messages with a given functionality based on their structure.

\noindent\textbf{run} a total order of events in a distributed system. This is what really hapend, but is unobservable due to clock skew, and an absent centralized clock.

Exampes of predicates which cannot be satisfied by a simple boolean predicate $A \vee B \wedge C$ where $A$, $B$, $C$ are variables.

\begin{itemize}
    \item Load ain the network is blaances.
    \item Resource allocation violates the \"at most k-out-of -n\" concurrent access requirement
    \item Message delay alone the route for A to B is less than 5ms
    \item No more than 100 total users logged in to tmachines A, B, and C
    \item In the computation of Fig 1 $x >= 2y$.
\end{itemize}

\noindent\textbf{sequence spec is an ordered array of simple predicates $SS ::= SP | SS; SP$}

Interval controlled sequences are a key component to realizing temporal
predicates esentially the process is as follows. An obervation is a set of
states $\Sigma$ on which a property can hold. A mentioned in the prior section
$SS ::= SP | SS; SP$ is a sequence of simple properties which hold over subsets
of an observation. Temporal properties can be constructed by linking
predicates. In the form $\phi;[\bar{\Theta}];\phi$ where all such phi are
observations on which a predicate holds.

$\phi_1;[\theta];\phi$ is satisfied by an observation if there exists a global
state $\Sigma^1$ in which $\phi_1$ holds, a late global state $\Sigma^2$ in
which $\phi_2$ holds, and $\Theta$ does not hold in the global states in
between. Denoted in the form $\phi_1;[\bar{\Theta}]\phi_2$.

This paper makes the contribution of formualting temporal strings of if
\textit{interval-constraied sequest} with the synyax
$\bar{[\Theta_1]}\phi_1;\dots;\bar{[\Theta_m]}\phi_m;\bar{[\Theta_{m+1}]}$

Key takeaway from the formulation. The difference between SS and ICS is that
between states $i_n,i_{n+1}$ there exists a $\bar{[\theta_n]}$ which must not
hold on that interval. In that way the expressivness of ICS is greater than
that of SS.

Two algorithms are proposed for solving SS \textbf{Pos} and \textbf{Def}. To
compute Def all paths through an observation are taken and must show that the
$SS$ formula holds. In constrast $SS$ \textbf{Pos} is verified simply by
finding a single example in an observation on which the formula holds.

\subsection{Take Away}

This paper presented a novel formulation for describing temporal properties in
distributed executions. It is very similar to \dinv in that it uses a latice to
formulate an execution. However, where \dinv prunes the search space by
examining only strongly consistant cuts, this approach dives in deeper, and
actually traverses all possible interleavings in order to satisfy a predicate.

If \dinv were to take this approach it would require that a user specify their
invariants ahead of time, if not then all possible observations of a run would
need to be fed into Daikon.

Of specific interest is their formulation of $\bar{[\theta]}\phi$ in that it
requirs some invariant to not hold before a know property in a given state.
Using \dinv's ground states this same technique could still be applied,
although it would be equally restricted by ground states. However, this could
still be a potential extension.

The clear problem with mining invariants using this technique is that it is a
search, therfore any attempt to infer invariants would result in inferring on
the entire search space.

This approach does not take into account failures, it could also be interesting
to extend the expressivness of the formula to accept failures and restarts. The
hardest part in this would be modifing $\Sigma$ to accept "Global" predicates
on a subset of nodes. Further it would be interesting to see if predicates
could be defined which allows for the recovery of a node, by a new one
introduced to the system.

Overall this is just a scheme for temporal logic detection in a system
instrumentd with vector clocks. It could be used for that golden key value spec
$\forall get(x) \leftarrow put(x) get(x) == get(x)$ untill $put(x')$. This would have been a cool path to go down if it had not allready been done.

\subsection{Questions}

\begin{itemize}
    \item Is there a way to divide temporal search paths for mining?
    \item Is there a way to identifiy identical search paths and merge them for compression?
       \item Is there an obvious or non obvious way to trim a search path, or limit the lenght of a search for temporal invariant mining?
       \item Can Search Paths be presented to daikon in a way which preserves temporal logic?
\end{itemize}




