\section{Incremental, Iterative Data Processing with Timely Dataflow \\ \small{ P Derek G. Murray, Frank McSherry, Micheal Isard, Rebecca Isaacs, Paul Barham, Martin Abadi}}

\subsection{notes}

This is a simplified version of the Naiad paper, but it is for public
consumption so perhaps it will be a good read none the less. It will also be
good to catch up on Naiad.

This paper was a bit of a waste of time to read, as it covered the details of
Naiad in less detail than the original paper.

Timely data flow is the key contribution to this work, although it was allready
presented in Naiad. The Idea is that distributed dataflow can be modeled as a
dataflow graph and itterations are possible by timestamping events, and using
partial ordering to determine if events are ready for processing.

One key advantage of Naiad is that it does not rely on the \textit{"think like
a vertex"} model of distributed computing. More robust algorithms are possible,
but are more complicated to implement than in a framework such as Pregal.

\subsection{observations}

\subsection{questions}

\begin{itemize}

\item Is Naiad the be all end all, what is needed to make distributed dataflow
more expresive? Is that extra expressiveness nessisary?

\item How bad is scheduling in Naiad in the worst case? Where does this
framework perform poorly.


\end{itemize}




