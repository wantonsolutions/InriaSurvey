\section{Disks for Data Centers \\ \small{Eric Brewer, Lawrence Ying, Lawrence
Greenfield, Robert Cypher, Theodore Ts'o}}

\subsection{notes}

This has nothing to do with anything really but it probably has some roots with
dataflow problems, and keeping data moving, so I'm going to read it.

The authors point out that disks for data centers do not require the same low
error rate that comodidy PC's require. This is due to the fact that all data
center data is nessisarily replicated globally. Reconstructing damaged data is
easy, so faster disks with higher capacity would be a boon at the loss of
accuracy.

A common practice for disk IO in a data center is to query many disks, and
return the value of the fastest read. Other reads are cancelled. The objective
to try and hit a cache, and gain minimal latency.

An interesting point by the authors is the want to allow disks to opperate
after head failures. They state that they can work around such failures, and it
would increase the lifespan of the disk. This is how you know you are working
at real scale.

\subsection{observations}

The authors of this paper really make one single point. Data centers want finer
grain control over disks. To rationalize this they use the end-to-end principle
as a basis. Essentially they are allready mannaging their own data, and have
very specific use cases for the disk, so many of the optimizations applied by
disk manufactures simply to not apply to them.

As stated above, the future of data center disks may be tall stacked, small
platters with inreliable IO.

Disk API's are a key concern for google practitoners. A potentially interesting
endevor would be to spec out an API for the modern disk, ie rather than flush
commands something like stream(start,finish,rate) would allow for per quality
streaming at a dedicated rate, while allowing for other types of IO to occur
concurently.

Another may be WriteBig(x, size) where a large portion of a disk is written
sequentailly in a part of the disk which can be sequentally read.

\subsection{questions}

\begin{itemize}

\item Are the projections of this paper true? Will disks remain relvent for the
next decade or will Flash take over. If disks are relevent, then for how long?

\item Do these companies actually know what they want? Dispite the apperant
knowledge of disks (at least to the point of their weak points and performance,
are companies such as Google, Facebook, and Amazon really ready to build all of
their systems for custom disks? The extra control may prove to cause larger
errors then they are ready for.

\end{itemize}
