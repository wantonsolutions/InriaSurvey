\section{On the Fly Testing of Regular Patterns in Distributed Computations \\
\small{Eddy Fromentin, Michel Raynal Vijay k Garg, Alex Tomlinson}}

\subsection{Notes}

This work aim to detect regular patterns in the exectuion of distrubted
systems. The first technique to note is that they reduced a distributed
execution down to only events corresponding to msg receves and sends. This is
the same intuition I had when initially building GoVector, and insturmenting
systems for \dinv but it is done formally.

Their technique relies on labling events with a symbol in an execution
language. Series of events form automata which are describable by regualr
expressions. Not sure where this paper is going, but I assume that they are
going to describe a partially orded regex syntax.

This paper did not go much further, the algorithm for performing on the fly
regular expressions is relativly simple. Essentally every process maintains an
array of the regular expression. The expression nessisarily begins at the
beginning of the computation, with omits much of the complexity incurred in the
\textit{specification and verification of dynamic properties in distributed
computation} paper. Each process maintains a boolean array of the properties in
the regex. When a new event occurs locally the process checks if it violates
the regex. Otherwise two seperate cases occur. 1) a sent message is received or
a local event happens, merge the two and check if they violate any of the regex
conditions. On a send event just treat it as a local event. In addition to
vector clocks the array of regular experession must be added to messages.

\subsection{Observations}

I think that regular expressions on states are interesting, but ultimatly not
going to play into my work. These models could be usefull for specifing
something like TCP where the execution is known to be finite, or should be. In
typical system cases though I believe that specifications need to be more
expressive than regular expersions. For instance Coq and TLA+ the key
languages in specification are fully fledged programming languages.

What this work does contribute is an online way to process regex. I think that
this is usefull for a couple of reasons 1) I think that it could be key in
generalizing some well known protocols such as leader election, where the set
of known messages would fit in the vocabulary of events.

I should do a bit of follow up on this work to see if anyone built a tool which
automatically translates regex into protocols rather than check a protocol with
regex, but I don't think that it would yeild interesting work.



