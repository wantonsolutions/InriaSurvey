\section{Shared Global states in Distrubted Computations \\ \small{Eddy
Fromentin, Michel Raynal}}

\subsection{Notes}

The paper begins with the same notation used in prior papers defining partial order and such. The key part of the paper is \textit{shared global states}, which are an element in the lattice of global states such that 

Let $\Sigma = (s_1,\dots,s_i,\dots,s_n) be a global state$ \\
$\Sigma is shared \iff $\\

$\forall (i,j)$ :: (prev($s_i$) $\rightarrow$ next($s_j$) \textit{or} $s_j =
s_j^{last} or s_i = s_i^0$)

Essentially a shared global state is a state which has the entire global
history as its transitive predisessor, or at least all of the history it could
known about transitivly.

The algorithm near the end of the paper shows how shared global state can be
detected on the fly, however the algorithm is centralized with diminishes how
interesting it is. That stated, shared global state is near conceptually with
common knowledge, in that it is the nth degree of eveyone knows everything, it
is the case that someone knows everything~\cite{Halpern:1990:KCK:79147.79161}.

\subsection{Observations}

While is is kind of cool that the idea of \textit{Shared Global States} has
been formalized here, there is a big want with reguard to purpose. Specifically
when would you want to use shared global state. I can only think of Chain
Replication.~\cite{vanRenesse:2004:CRS:1251254.1251261}. While shared global
state is not a central tenant of the system it would acheive it quite often
during execution.

With reguard to Dinv this is almost certanly not the path to go down, SCC are
allready too much of a restriction as they do no even consider the world of
\textit{observations}. With that said GSS could be an interestin root at which
to perform some sort of analysis. It serves as an obvious break point.

If I wanted to infer arbetrary temporal properties on a distributed execution I
could for instance make users provide a template, search for observations from
GSS roots, enumerate them in a key value store, overlay matching or equivalent
states (there would be few, but some may match the template.

This is probably not the way to go either, the size of the output would be
gigantic, just note that \textbf{GSS can be used for roots of computation} as
they are as close to common knowledge as you are ever going to get (and it can
be done on the fly)

\subsection{Questions}

\begin{itemize}

\item Is shared global state sufficient to coordinate locks? If not what
information is required to allow for locking?

\item How frequently do these happen in practice, for example in etcd? If the
answer is kind of frequently there may be some interesting work to be done.

\end{itemize}




